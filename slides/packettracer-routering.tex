\section{Routering in een internet.}

\begin{frame}{Hoeveel netwerken tel je?}
\includegraphics[height=\textheight]{img/pt-hoeveel-netwerken.png}
\end{frame}

\begin{frame}{Hoeveel netwerken tel je?}
\includegraphics[width=\textwidth]{img/pt-hoeveel-netwerken-2.png}
\end{frame}


\begin{frame}{Routering.}
\begin{itemize}
  \item Elke host/router heeft een \alert{routetabel}
  \item Een routetabel bevat voor \textit{elke mogelijke bestemming} de \alert{eerstvolgende} stap
  \begin{itemize}
    \item Rechtstreeks verbonden netwerken: meteen afleveren
    \item Statische of dynamische routes
    \item Standaardroute, \textit{default gateway}
  \end{itemize}
\end{itemize}
\end{frame}

\begin{frame}{Routeringsprotocols.}
\begin{itemize}
  \item Routers wisselen informatie uit over de te volgen routes
  \begin{itemize}
    \item vb.~Router Information Protocol (RIP)
  \end{itemize}
  \item Nieuwe routes worden automatisch toegevoegd aan de routetabel
\end{itemize}

\centering
\includegraphics[width=.5\textwidth]{img/ip-routering}

\end{frame}

\begin{frame}[fragile]
\frametitle{Routetabel opvragen (Linux).}

\scriptsize
\begin{Verbatim}[commandchars=\\\{\}]
$ \alert{ip route}
default via 192.168.0.1 dev wlp2s0 proto dhcp metric 600 
192.168.0.0/24 dev wlp2s0 proto kernel scope link src 192.168.0.185 metric 600 
192.168.122.0/24 dev virbr0 proto kernel scope link src 192.168.122.1 linkdown 
\end{Verbatim}

\normalsize Op Windows: \alert{\texttt{route print}}

\end{frame}


\begin{frame}[fragile]
\frametitle{Routetabel opvragen (Linux).}

\scriptsize
\begin{Verbatim}[commandchars=\\\{\}]
$ ip route
\alert{default via 192.168.0.1} dev wlp2s0 proto dhcp metric 600 
192.168.0.0/24 dev wlp2s0 proto kernel scope link src 192.168.0.185 metric 600 
192.168.122.0/24 dev virbr0 proto kernel scope link src 192.168.122.1 linkdown 
\end{Verbatim}

\normalsize Default gateway

\end{frame}

\begin{frame}[fragile]
\frametitle{Routetabel opvragen (Linux).}

\scriptsize
\begin{Verbatim}[commandchars=\\\{\}]
$ ip route
default via 192.168.0.1 dev wlp2s0 proto dhcp metric 600 
\alert{192.168.0.0/24} dev wlp2s0 proto kernel scope link src 192.168.0.185 metric 600 
\alert{192.168.122.0/24} dev virbr0 proto kernel scope link src 192.168.122.1 linkdown 
\end{Verbatim}

\normalsize Rechtstreeks verbonden netwerken: meteen afleveren

\end{frame}

\begin{frame}[fragile]
\frametitle{Routetabel opvragen (Cisco).}

\scriptsize
\begin{Verbatim}[commandchars=\\\{\}]
Router>\alert{show ip route}
Codes: C - connected, S - static, I - IGRP, R - RIP, M - mobile, B - BGP
D - EIGRP, EX - EIGRP external, O - OSPF, IA - OSPF inter area
N1 - OSPF NSSA external type 1, N2 - OSPF NSSA external type 2
E1 - OSPF external type 1, E2 - OSPF external type 2, E - EGP
i - IS-IS, L1 - IS-IS level-1, L2 - IS-IS level-2, ia - IS-IS inter area
* - candidate default, U - per-user static route, o - ODR
P - periodic downloaded static route

Gateway of last resort is not set

C    192.168.1.0/24 is directly connected, FastEthernet0/0
R    192.168.2.0/24 [120/1] via 192.168.3.2, 00:00:11, Serial0/0/0
C    192.168.3.0/24 is directly connected, Serial0/0/0
\end{Verbatim}

\end{frame}

\begin{frame}[fragile]
\frametitle{Routetabel opvragen (Cisco).}

\scriptsize
\begin{Verbatim}[commandchars=\\\{\}]
Router>show ip route
Codes: C - connected, S - static, I - IGRP, R - RIP, M - mobile, B - BGP
D - EIGRP, EX - EIGRP external, O - OSPF, IA - OSPF inter area
N1 - OSPF NSSA external type 1, N2 - OSPF NSSA external type 2
E1 - OSPF external type 1, E2 - OSPF external type 2, E - EGP
i - IS-IS, L1 - IS-IS level-1, L2 - IS-IS level-2, ia - IS-IS inter area
* - candidate default, U - per-user static route, o - ODR
P - periodic downloaded static route

\alert{Gateway of last resort is not set}

C    192.168.1.0/24 is directly connected, FastEthernet0/0
R    192.168.2.0/24 [120/1] via 192.168.3.2, 00:00:11, Serial0/0/0
C    192.168.3.0/24 is directly connected, Serial0/0/0
\end{Verbatim}

\normalsize Geen default gateway

\end{frame}

\begin{frame}[fragile]
\frametitle{Routetabel opvragen (Cisco).}

\scriptsize
\begin{Verbatim}[commandchars=\\\{\}]
Router>show ip route
Codes: \alert{C - connected}, S - static, I - IGRP, R - RIP, M - mobile, B - BGP
D - EIGRP, EX - EIGRP external, O - OSPF, IA - OSPF inter area
N1 - OSPF NSSA external type 1, N2 - OSPF NSSA external type 2
E1 - OSPF external type 1, E2 - OSPF external type 2, E - EGP
i - IS-IS, L1 - IS-IS level-1, L2 - IS-IS level-2, ia - IS-IS inter area
* - candidate default, U - per-user static route, o - ODR
P - periodic downloaded static route

Gateway of last resort is not set

\alert{C    192.168.1.0/24} is directly connected, FastEthernet0/0
R    192.168.2.0/24 [120/1] via 192.168.3.2, 00:00:11, Serial0/0/0
\alert{C    192.168.3.0/24} is directly connected, Serial0/0/0
\end{Verbatim}

\normalsize Rechtstreeks verbonden netwerken

\end{frame}

\begin{frame}[fragile]
\frametitle{Routetabel opvragen (Cisco).}

\scriptsize
\begin{Verbatim}[commandchars=\\\{\}]
Router>show ip route
Codes: C - connected, S - static, I - IGRP, \alert{R - RIP}, M - mobile, B - BGP
D - EIGRP, EX - EIGRP external, O - OSPF, IA - OSPF inter area
N1 - OSPF NSSA external type 1, N2 - OSPF NSSA external type 2
E1 - OSPF external type 1, E2 - OSPF external type 2, E - EGP
i - IS-IS, L1 - IS-IS level-1, L2 - IS-IS level-2, ia - IS-IS inter area
* - candidate default, U - per-user static route, o - ODR
P - periodic downloaded static route

Gateway of last resort is not set

C    192.168.1.0/24 is directly connected, FastEthernet0/0
\alert{R}    192.168.2.0/24 [120/1] via 192.168.3.2, 00:00:11, Serial0/0/0
C    192.168.3.0/24 is directly connected, Serial0/0/0
\end{Verbatim}

\normalsize Automatisch toegevoegd via Router Information Protocol (RIP)

\end{frame}


\begin{frame}[fragile]
\frametitle{Routetabel opvragen (Cisco).}

\scriptsize
\begin{Verbatim}[commandchars=\\\{\}]
Router>show ip route
Codes: C - connected, S - static, I - IGRP, R - RIP, M - mobile, B - BGP
D - EIGRP, EX - EIGRP external, O - OSPF, IA - OSPF inter area
N1 - OSPF NSSA external type 1, N2 - OSPF NSSA external type 2
E1 - OSPF external type 1, E2 - OSPF external type 2, E - EGP
i - IS-IS, L1 - IS-IS level-1, L2 - IS-IS level-2, ia - IS-IS inter area
* - candidate default, U - per-user static route, o - ODR
P - periodic downloaded static route

Gateway of last resort is not set

C    192.168.1.0/24 is directly connected, FastEthernet0/0
R    \alert{192.168.2.0/24} [120/1] \alert{via 192.168.3.2}, 00:00:11, Serial0/0/0
C    192.168.3.0/24 is directly connected, Serial0/0/0
\end{Verbatim}

\normalsize Route naar ander netwerk via router (next hop)

\end{frame}


\begin{frame}{Vervolg PacketTracer-oefening.}
  \centering
  \includegraphics[height=.7\textheight]{img/pt-routing}
\end{frame}

\begin{frame}{Vervolg PacketTracer-oefening.}
  \begin{itemize}
    \item Stel de \alert{routers} in met RIP
    \item Vul telkens de netwerken in waarmee de router \alert{rechtstreeks verbonden} is
    \item Alle hosts moeten een \alert{Gateway} hebben en liefst ook een DNS server
  \end{itemize}
\end{frame}

