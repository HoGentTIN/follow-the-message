\section{De opleiding toegepaste informatica.}

\begin{frame}[plain]

  {\huge \textbf{Wie overweegt \alert{informatica} te gaan studeren?}}

  \pause
  \bigskip
  \bigskip

  {\Large Waarom (niet)?}

\end{frame}

\begin{frame}[plain]
\centering
\includegraphics[height=\paperheight]{img/nerd.png}
\end{frame}

\begin{frame}[plain]
\centering
\includegraphics[width=.9\paperwidth]{img/lara.png}
\end{frame}

\begin{frame}[plain]
\centering
\includegraphics[height=\paperheight]{img/it-job-titles.png}
\end{frame}

\begin{frame}[plain]

\frametitle{Hoeveel verdient een IT'er?}

\begin{table}
  \begin{tabular}{lr}
  	\toprule
  	\textbf{Jobtitel}   & \textbf{Salaris (\euro)} \\
  	\midrule
  	IT Director         &            7 000 \\
  	IT Manager          &            5 300 \\
  	Projectmanager      &            4 500 \\
  	Network Engineer    &            4 100 \\
  	Business Analyst    &            3 700 \\
  	Functioneel Analyst &            3 600 \\
  	System Engineer     &            3 600 \\
  	Software Developer  &            3 600 \\
  	Support Engineer    &            2 900 \\
  	\bottomrule
  \end{tabular}
  
  \label{tab:lonen-it}
  \caption{Mediaan bruto maandloon. Bron: Robert Half, Salarisgids België 2019}
\end{table}

\end{frame}

\begin{frame}{Toegepaste informatica @HOGENT.}
\centering
\includegraphics[height=.8\textheight]{img/t-shape.png}
\end{frame}

\begin{frame}[plain]
\centering
\includegraphics[height=\paperheight]{img/tin-leerlijnen.png}
\end{frame}