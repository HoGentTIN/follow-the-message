\section{Netwerksimulatie met PacketTracer.}

\begin{frame}{Een eenvoudig LAN opzetten.}

\begin{itemize}
  \item Voeg een \alert{switch}, \alert{PC} en \alert{server} toe
  \item \alert{Verbind} beide computers met de switch
  \begin{itemize}
    \item Copper Straight-Through kabel
  \end{itemize}
  \item Geef beide computers een \alert{IP-adres}
  \begin{itemize}
    \item binnen hetzelfde IP-netwerk!
  \end{itemize}
  \item Testen
  \begin{itemize}
    \item Test verbinding met \alert{\texttt{ping}}
    \item Kan je vanaf de PC de website op de server zien?
  \end{itemize}
\end{itemize}

\centering
\includegraphics[width=7cm]{img/pt-pc-server.png}

\end{frame}

\begin{frame}{Netwerkdiensten (services).}
  \Large\bfseries
  \alert{Applicaties} die over een netwerk communiceren volgens een vastgelegde ``taal'' of \alert{protocol}.
\end{frame}

\begin{frame}{Netwerkdiensten (services).}
\centering
\includegraphics[width=.9\textwidth]{img/protocols.png}
\end{frame}

\begin{frame}{Poortnummers.}
\centering
\begin{tabular}{cll}
	\toprule
	\textbf{Poortnr.} & \textbf{Protocol} & \textbf{Definitie}               \\
	\midrule
	20                & FTP data          & File Transfer Protocol           \\
	21                & FTP control       &                                  \\
	22                & SSH               & Secure SHell                     \\
	25                & SMTP              & Simple Mail Transfer Protocol    \\
	53                & DNS               & Domain Name Service              \\
	80                & HTTP              & Hypertext Transfer Protocol      \\
	143               & IMAP              & Internet Message Access Protocol \\
	443               & HTTPS             & HTTP Secure                      \\
	\bottomrule
\end{tabular}
\end{frame}

\begin{frame}{Poortnummers.}
  \begin{itemize}
    \item \alert{0-1023} Well known ports, systeempoorten
    \item \alert{1024-49151} Registered ports
    \item \alert{49152-65535} Dynamic ports, clients
  \end{itemize}

  Zie: \url{https://en.wikipedia.org/wiki/List_of_TCP_and_UDP_port_numbers}
\end{frame}

\begin{frame}{Adressering.}
\framesubtitle{Het hele verhaal.}
\large\centering
\begin{tabular}{|c|l}
  \cline{1-1}
  Applicatie     & \\
  \cline{1-1}
  Transport      & $\leftarrow$ Poortnummer \\
  \cline{1-1}
  Internet       & $\leftarrow$ IP-adres \\
  \cline{1-1}
  Netwerktoegang & $\leftarrow$ MAC-adres \\
  \cline{1-1}
  Hardware       & \\
  \cline{1-1}
\end{tabular}
\end{frame}

\begin{frame}[plain]
\frametitle{De structuur van een IP-pakket.}
\centering
\includegraphics[height=.9\textheight]{img/ip-pakket-structuur}
\end{frame}

\begin{frame}[plain]
\frametitle{Adressering in een IP-pakket.}
\centering
\includegraphics[height=.9\textheight]{img/ip-pakket-adressering}
\end{frame}

\begin{frame}{Een service toevoegen.}
\begin{itemize}
  \item Activeer DNS service op de server
  \item Geef de server een naam, bv. \alert{www.workshop.net}
  \item Zorg dat de pc's de server als DNS gebruiken
  \item Kan je de website bekijken via \alert{http://www.workshop.net/}?
\end{itemize}
\end{frame}