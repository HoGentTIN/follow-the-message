\section{Adressering.}

\begin{frame}{TCP/IP protocolstack}
\centering
\large
\begin{tabular}{|c|}
  \hline
  Applicatie     \\ \hline
  Transport      \\ \hline
  Internet       \\ \hline
  Netwerktoegang \\ \hline
  Hardware       \\
  \hline
\end{tabular}
\end{frame}

\begin{frame}
  \begin{figure}
    \centering
    \includegraphics[height=.8\textheight]{img/NIC-mac-HP.jpg}
    
    \caption{Fysieke adressering: MAC-adres (Media Access Control), vb. ``00:02:a5:44:2a:30''}
  \end{figure}
\end{frame}

\begin{frame}
  \begin{figure}
    \centering
  
    \includegraphics[height=.8\textheight]{img/ip-instellingen-windows.png}
    
    \caption{Logische adressering: IPv4-adres (Internet Protocol, versie 4), vb. ``192.168.1.2''}
  \end{figure}
\end{frame}

\begin{frame}[fragile]
  \frametitle{Adressen opvragen (Linux).}
  \scriptsize
\begin{Verbatim}[commandchars=\\\{\}]
$ \alert{ip addr}
1: lo: <LOOPBACK,UP,LOWER_UP> mtu 65536 qdisc noqueue state ...
    link/loopback 00:00:00:00:00:00 brd 00:00:00:00:00:00
    inet 127.0.0.1/8 scope host lo
       valid_lft forever preferred_lft forever
    inet6 ::1/128 scope host 
       valid_lft forever preferred_lft forever
2: wlp3s0: <BROADCAST,MULTICAST,UP,LOWER_UP> mtu 1500 qdisc ...
    link/ether 0c:84:dc:8e:82:cc brd ff:ff:ff:ff:ff:ff
    inet 192.168.10.244/24 brd 192.168.10.255 scope global ...
       valid_lft 75631sec preferred_lft 75631sec
    inet6 fe80::e84:dcff:fe8e:82cc/64 scope link 
       valid_lft forever preferred_lft forever
3: em1: <NO-CARRIER,BROADCAST,MULTICAST,UP> mtu 1500 qdisc ...
    link/ether f0:1f:af:2a:62:32 brd ff:ff:ff:ff:ff:ff
\end{Verbatim}
\end{frame}

\begin{frame}[fragile]
  \frametitle{Adressen opvragen (Linux).}
  \scriptsize
\begin{Verbatim}[commandchars=\\\{\}]
$ ip addr
1: lo: <LOOPBACK,UP,LOWER_UP> mtu 65536 qdisc noqueue state ...
    link/loopback \alert{00:00:00:00:00:00} brd 00:00:00:00:00:00
    inet 127.0.0.1/8 scope host lo
       valid_lft forever preferred_lft forever
    inet6 ::1/128 scope host 
       valid_lft forever preferred_lft forever
2: wlp3s0: <BROADCAST,MULTICAST,UP,LOWER_UP> mtu 1500 qdisc ...
    link/ether \alert{0c:84:dc:8e:82:cc} brd ff:ff:ff:ff:ff:ff
    inet 192.168.10.244/24 brd 192.168.10.255 scope global ...
       valid_lft 75631sec preferred_lft 75631sec
    inet6 fe80::e84:dcff:fe8e:82cc/64 scope link 
       valid_lft forever preferred_lft forever
3: em1: <NO-CARRIER,BROADCAST,MULTICAST,UP> mtu 1500 qdisc ...
    link/ether f0:1f:af:2a:62:32 brd ff:ff:ff:ff:ff:ff
\end{Verbatim}

\hspace{2cm}\textbf{Hardware (MAC) adres}

\end{frame}

\begin{frame}[fragile]
\frametitle{Adressen opvragen (Linux).}
\scriptsize
\begin{Verbatim}[commandchars=\\\{\}]
$ ip addr
1: lo: <LOOPBACK,UP,LOWER_UP> mtu 65536 qdisc noqueue state ...
    link/loopback 00:00:00:00:00:00 brd 00:00:00:00:00:00
    inet \alert{127.0.0.1/8} scope host lo
       valid_lft forever preferred_lft forever
    inet6 ::1/128 scope host 
       valid_lft forever preferred_lft forever
2: wlp3s0: <BROADCAST,MULTICAST,UP,LOWER_UP> mtu 1500 qdisc ...
    link/ether 0c:84:dc:8e:82:cc brd ff:ff:ff:ff:ff:ff
    inet \alert{192.168.10.244/24} brd 192.168.10.255 scope global ...
       valid_lft 75631sec preferred_lft 75631sec
    inet6 fe80::e84:dcff:fe8e:82cc/64 scope link 
       valid_lft forever preferred_lft forever
3: em1: <NO-CARRIER,BROADCAST,MULTICAST,UP> mtu 1500 qdisc ...
    link/ether f0:1f:af:2a:62:32 brd ff:ff:ff:ff:ff:ff
\end{Verbatim}

\hspace{2cm}\textbf{IPv4-adres}

\end{frame}

\begin{frame}[fragile]
\frametitle{Adressen opvragen (Linux).}
\scriptsize
\begin{Verbatim}[commandchars=\\\{\}]
$ ip addr
1: lo: <LOOPBACK,UP,LOWER_UP> mtu 65536 qdisc noqueue state ...
    link/loopback 00:00:00:00:00:00 brd 00:00:00:00:00:00
    inet 127.0.0.1/8 scope host lo
       valid_lft forever preferred_lft forever
    inet6 \alert{::1/128} scope host 
       valid_lft forever preferred_lft forever
2: wlp3s0: <BROADCAST,MULTICAST,UP,LOWER_UP> mtu 1500 qdisc ...
    link/ether 0c:84:dc:8e:82:cc brd ff:ff:ff:ff:ff:ff
    inet 192.168.10.244/24 brd 192.168.10.255 scope global ...
       valid_lft 75631sec preferred_lft 75631sec
    inet6 \alert{fe80::e84:dcff:fe8e:82cc/64} scope link 
       valid_lft forever preferred_lft forever
3: em1: <NO-CARRIER,BROADCAST,MULTICAST,UP> mtu 1500 qdisc ...
    link/ether f0:1f:af:2a:62:32 brd ff:ff:ff:ff:ff:ff
\end{Verbatim}

\hspace{2cm}\textbf{IPv6-adres}

\end{frame}

\begin{frame}{Adressen opvragen (Windows).}
\begin{itemize}
  \item Probeer dit zelf!
  \item Open een terminalvenster (Win+R, \texttt{cmd})
  \item Commando \textcolor{hgblue}{\texttt{ipconfig}} of \textcolor{hgblue}{\texttt{ipconfig /all}}
  \begin{itemize}
    \item Wat is het MAC adres?
    \item Wat is het IP adres?
  \end{itemize}
\end{itemize}
\end{frame}

\begin{frame}{Fysieke (MAC) vs logische (IP) adressering.}
\centering
\begin{tabular}{ll}
	\toprule
	\textbf{MAC}               & \textbf{IP}              \\
	\midrule
	Hardware                   & Software                 \\
	Afh.~technologie           & Universeel               \\
	Elke NIC heeft uniek adres & Elke host binnen LAN     \\
	                           & heeft uniek adres        \\
	Enkel binnen LAN           & Internet/WAN             \\
	vgl.~naam (onveranderlijk) & vgl~adres (veranderlijk, \\
	                           & bv.~verhuis)             \\
	\bottomrule
\end{tabular}
\end{frame}

\begin{frame}{Switch vs router.}
\begin{columns}
  \column{.5\textwidth}
  
  \includegraphics[width=\textwidth]{img/switch-netgear.png}
  
  \textbf{Switch}
  
  \begin{itemize}
    \item Kent enkel \alert{MAC} adressen
    \item Werkt enkel op \alert{Netwerktoegang}slaag
  \end{itemize}
  
  \column{.5\textwidth}
  
  \includegraphics[width=\textwidth]{img/router-cisco.png}
  
  \textbf{Router}
  
  \begin{itemize}
    \item Kent ook \alert{IP}-adressen
    \item Werkt op de \alert{Internet}laag
  \end{itemize}
\end{columns}
\end{frame}

\begin{frame}{TCP/IP protocolstack}
\centering
\large
\begin{tabular}{|c|}
  \hline
  Applicatie     \\ \hline
  Transport      \\ \hline
  Internet       \\ \hline
  Netwerktoegang \\ \hline
  Hardware       \\
  \hline
\end{tabular}
\end{frame}

\begin{frame}[plain]
\frametitle{Structuur van een IP-adres.}
\centering
\includegraphics[height=.9\textheight]{img/ip-network-host.jpg}
\end{frame}

\begin{frame}[plain]
\frametitle{Netwerkmasker.}
\centering
\includegraphics[height=.9\textheight]{img/ip-network-mask.png}
\end{frame}
